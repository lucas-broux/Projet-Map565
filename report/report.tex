\documentclass[11pt,letterpaper]{article}
\usepackage[utf8]{inputenc}
\usepackage[french]{babel}
\usepackage{amsmath}
\usepackage{mathtools}
\usepackage{amsfonts}
\usepackage{dsfont}
\usepackage{amssymb}
\usepackage{graphicx}
\usepackage{amsthm}
\usepackage{hyperref}
\usepackage{url}
\usepackage{float}
\usepackage{siunitx}
\usepackage[left=3cm,right=3cm,top=3cm,bottom=3cm]{geometry}
\usepackage{makeidx}
\makeatletter
\newcommand{\mathleft}{\@fleqntrue\@mathmargin0pt}
\newcommand{\mathcenter}{\@fleqnfalse}
\makeatother

\makeindex

% Define 2 graphic paths to images.
\graphicspath{{images/}{../images/}}

% Import subfiles package. 
\usepackage{subfiles}

% Define new theorem style.
\newtheorem{prop}{Proposition}[section]


\begin{document}

%\begin{center}
%\includegraphics[scale=0.2]{../../../Desktop/logo.jpg}
%\end{center}

% Front page.
\subfile{sections/titlepage}

% Page of contents.
\pagebreak
\tableofcontents

% Introduction.
\pagebreak
\section{Introduction}
\subfile{sections/introduction}

% Première partie : approche "série temporelle".
\pagebreak
\section{Analyse de la série temporelle}
\subfile{sections/section_1}

% Deuxième partie : statistique des extrêmes.
\pagebreak
\section{Statistique des extrêmes}
\subfile{sections/section_2}

% Troisième partie : modélisation des dépendances.
\pagebreak
\section{Modélisation des dépendances}
\subfile{sections/section_3}


% Conclusion
\pagebreak
\section{Conclusion}
\subfile{sections/conclusion}


% Bibliography.
\pagebreak
\nocite{*}
\bibliographystyle{plain}
\bibliography{projectbib}


\end{document}