\documentclass[../report.tex]{subfiles}

\begin{document}

% Introduction du sujet.
\par Le but de ce projet est d'appliquer les méthodes étudiées en cours de modélisation statistique (MAP565) à l'étude de l'évolution du climat sur plusieurs villes de France.

% Motivation.
\par Cette analyse est motivée par plusieurs facteurs. D'une part, le réchauffement climatique est un phénomène avéré de ces dernières décennies; il convient donc de le surveiller et l'analyser avec des méthodes de modélisation plus ou moins développées. D'autre part, certaines stations météorologiques relèvent des données de manière journalières depuis des dizaines d'années et distribuent leurs bases de données sur internet. Enfin, les méthodes vues en cours nous semblent particulièrement pertinentes dans ce cadre.

% Problématiques abordées.
\par Nous abordons cette étude sous un angle triple.
Dans un premier temps, nous emploierons une approche de type "série temporelle" afin de modéliser l'évolution de la température à Bordeaux, modélisation que nous testerons en estimant par prédiction les valeurs obtenues en 2017.
Dans une deuxième partie, nous nous intéresserons à la statistique des extrêmes de cette même série temporelle, afin de proposer une analyse des risques de canicules.
Enfin, nous utiliserons des méthodes de modélisation de dépendances afin de déterminer si les risques de canicules à Paris et à Bordeaux sont liés.

% Données, bibliothèques.
\par Nous utilisons les données fournies par le site internet \url{https://www.ecad.eu/}, que nous avons traitées et converties en format .csv afin de pouvoir les étudier. 
Nous implémentons nos scripts en Python et en R.
Toutes les données, ainsi que les scripts utilisés, sont disponibles sur le repository github du projet : \url{https://github.com/lucas-broux/Projet-Map565}.

\end{document}